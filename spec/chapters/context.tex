\chapter{Context}
\label{context}

\jaxrs\ provides facilities for obtaining and processing information about the application deployment context and the context of individual requests. Such information is available to both resource classes (see chapter \ref{resources}) and providers (see chapter \ref{providers}). This chapter describes these facilities.

\section{URIs and URI Templates}

An instance of \UriInfo\ can be injected into a class field or method parameter using the \Context\ annotation. \UriInfo\ provides both static and dynamic, per-request information, about the components of a request URI. E.g. the following would return the names of any query parameters in a request:

\begin{listing}{1}
@HttpMethod(GET)
@ProduceMime{"text/plain"}
public String listQueryParamNames(@Context UriInfo info) {
  StringBuilder buf = new StringBuilder();
  for (String param: info.getQueryParameters().keySet()) {
    buf.append(param);
    buf.append("\n");
  }
  return buf.toString();
}
\end{listing} 

\section{Headers}

An instance of \HttpHeaders\ can be injected into a class field or method parameter using the \Context\ annotation. \HttpHeaders\ provides access to request header information either in map form or via strongly typed convenience methods. E.g. the following would return the names of all the headers in a request:

\begin{listing}{1}
@HttpMethod(GET)
@ProduceMime{"text/plain"}
public String listHeaderNames(@Context HttpHeaders headers) {
  StringBuilder buf = new StringBuilder();
  for (String header: headers.getRequestHeaders().keySet()) {
    buf.append(header);
    buf.append("\n");
  }
  return buf.toString();
}
\end{listing}

Note that response headers may be provided using the \Response\ interface, see \ref{resource_method_return} for more details.

\section{Content Negotiation and Preconditions}

\jaxrs\ simplifies support for content negotiation and preconditions using the \Request\ interface. An instance of \Request\ can be injected into a class field or method parameter using the \Context\ annotation. The methods of \Request\ allow a caller to determine the best matching representation variant and to evaluate whether the current state of the resource matches any preconditions in the request. Precondition support methods return a \Response\ that can be returned to the client to inform it that the request preconditions were not met. E.g. the following checks if the current entity tag matches any preconditions in the request before updating the resource:

\begin{listing}{1}
@HttpMethod(PUT)
public Response updateFoo(@Context Request request, Foo foo) {
	EntityTag tag = getCurrentTag();
	Response response = request.evaluate(tag, null);
	if (response != null)
	  return response;
	else
	  return doUpdate(foo);
}
\end{listing}

\section{Security Context}

The \SecurityContext\ interface provides access to information about the security context of the current request. An instance of \SecurityContext\ can be injected into a class field or method parameter using the \Context\ annotation. The methods of \SecurityContext\ provide access to the current user principle, information about roles assumed by the requester, whether the request arrived over a secure channel and the authentication scheme used.

\section{Injection Scope}

When the \Context\ annotation is applied to a class field, an implementation is only required to inject the applicable context into those class instances created by the implementation runtime. Objects returned by sub-resource locators (see section \ref{sub_resources}) are expected to be initialized by their creator and are not subject to resource injection by the implementation runtime.
